\section{Introduction}
\label{sec:intro}

Scalability is often heralded as the most important property of a ``big data''
software system. In~\cite{McSherryCOST}, McSherry et al. show that this metric
is frequently misleading when not coupled with performance comparisons against
an appropriate baseline. Providing decreased runtime as core count increases is
certainly a desirable property, but requiring orders of magnitude more hardware
to achieve the same performance as a single thread is very much not.

McSherry et al. introduce a metric they term COST, or \textbf{C}onfiguration
that \textbf{O}utperforms a \textbf{S}ingle \textbf{T}hread. This metric
measures the amount of hardware required before a particular system beats the
performance of a reasonably designed singled threaded solution.
