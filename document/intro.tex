\section{Introduction}
\label{sec:intro}

Scalability is often heralded as the most important property of a ``big data''
software system. In~\cite{McSherryCOST}, \citeauthor{McSherryCOST} show that
this metric is frequently misleading when not coupled with performance
comparisons against an appropriate baseline. Providing decreased runtime as core
count increases is certainly a desirable property of a well-engineered parallel
system, but in the end, absolute performance is paramount.

\citeauthor{McSherryCOST} introduce a metric they term COST, or
\textbf{C}onfiguration that \textbf{O}utperforms a \textbf{S}ingle
\textbf{T}hread. This metric measures the amount of hardware required before a
particular system beats the performance of a reasonably designed single threaded
solution. In particular, they found that several parallel data processing
systems from the literature were outperformed by a simple singled threaded
program for various common graph processing tasks, even when the parallel
systems were given over 100x the computational resources.


% but inducing requiring orders of
% magnitude more hardware to achieve the same performance as a single thread is
% very much not.
