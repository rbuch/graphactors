    %%%%%%%%%%%%%%%%%%%%%%%%%%%%%%%%%%%%%%%%%
% Journal Article
% LaTeX Template
% Version 1.4 (15/5/16)
%
% This template has been downloaded from:
% http://www.LaTeXTemplates.com
%
% Original author:
% Frits Wenneker (http://www.howtotex.com) with extensive modifications by
% Vel (vel@LaTeXTemplates.com)
%
% License:
% CC BY-NC-SA 3.0 (http://creativecommons.org/licenses/by-nc-sa/3.0/)
%
%%%%%%%%%%%%%%%%%%%%%%%%%%%%%%%%%%%%%%%%%

%----------------------------------------------------------------------------------------
%	PACKAGES AND OTHER DOCUMENT CONFIGURATIONS
%----------------------------------------------------------------------------------------

\documentclass[twoside,twocolumn]{article}

\usepackage{blindtext} % Package to generate dummy text throughout this template

\usepackage[sc]{mathpazo} % Use the Palatino font
\usepackage[T1]{fontenc} % Use 8-bit encoding that has 256 glyphs
\linespread{1.05} % Line spacing - Palatino needs more space between lines
\usepackage{microtype} % Slightly tweak font spacing for aesthetics

\usepackage[english]{babel} % Language hyphenation and typographical rules

\usepackage[hmarginratio=1:1,top=32mm,columnsep=20pt]{geometry} % Document margins
\usepackage[hang, small,labelfont=bf,up,textfont=it,up]{caption} % Custom captions under/above floats in tables or figures
\usepackage{booktabs} % Horizontal rules in tables

\usepackage{lettrine} % The lettrine is the first enlarged letter at the beginning of the text

\usepackage{enumitem} % Customized lists
\setlist[itemize]{noitemsep} % Make itemize lists more compact

\usepackage{abstract} % Allows abstract customization
\renewcommand{\abstractnamefont}{\normalfont\bfseries} % Set the "Abstract" text to bold
\renewcommand{\abstracttextfont}{\normalfont\small\itshape} % Set the abstract itself to small italic text

\usepackage{titlesec} % Allows customization of titles
\renewcommand\thesection{\Roman{section}} % Roman numerals for the sections
\renewcommand\thesubsection{\roman{subsection}} % roman numerals for subsections
\titleformat{\section}[block]{\large\scshape\centering}{\thesection.}{1em}{} % Change the look of the section titles
\titleformat{\subsection}[block]{\large}{\thesubsection.}{1em}{} % Change the look of the section titles

\usepackage{fancyhdr} % Headers and footers
\pagestyle{fancy} % All pages have headers and footers
\fancyhead{} % Blank out the default header
\fancyfoot{} % Blank out the default footer
\fancyhead[C]{COST of Graph Processing Using Actors} % Custom header text
\fancyfoot[C]{\thepage} % Custom footer text

\usepackage{titling} % Customizing the title section

\usepackage{hyperref} % For hyperlinks in the PDF

% My customizations:
\usepackage[numbers]{natbib}
%\usepackage{parskip}

%----------------------------------------------------------------------------------------
%	TITLE SECTION
%----------------------------------------------------------------------------------------

\setlength{\droptitle}{-4\baselineskip} % Move the title up

\pretitle{\begin{center}\Huge\bfseries} % Article title formatting
\posttitle{\end{center}} % Article title closing formatting
\title{COST of Graph Processing Using Actors}
\author{%
\textsc{Ronak Buch}\\[1ex] % Your name
\normalsize University of Illinois at Urbana-Champaign \\ % Your institution
\normalsize \href{mailto:rabuch2@illinois.edu}{rabuch2@illinois.edu} % Your email address
%\and % Uncomment if 2 authors are required, duplicate these 4 lines if more
%\textsc{Jane Smith}\thanks{Corresponding author} \\[1ex] % Second author's name
%\normalsize University of Utah \\ % Second author's institution
%\normalsize \href{mailto:jane@smith.com}{jane@smith.com} % Second author's email address
}
\date{} % Leave empty to omit a date
\renewcommand{\maketitlehookd}{%
\begin{abstract}
  Graph processing is an increasingly important domain of computer science, with
  applications in data and network analysis, among others. Target graphs in
  these applications are often large, leading to the creating of ``big data''
  systems designed to provide the scalability needed to analyze these graphs
  using parallelism. However, researchers have shown that while these systems
  often provide scalability, they also often introduce overheads that exceed the
  benefits they provide, with lower absolute performance than simple serial
  implementations.

  This report studies the viability and performance of using the actor model to
  perform common graph tasks. We show that relatively simple actor-based
  implementations outperform both dedicated graph processing systems and the
  benchmark serial code.
\end{abstract}

}

%----------------------------------------------------------------------------------------

\begin{document}

% Print the title
\maketitle

%----------------------------------------------------------------------------------------
%	ARTICLE CONTENTS
%----------------------------------------------------------------------------------------

\section{Introduction}
\label{sec:intro}

Scalability is often heralded as the most important property of a ``big data''
software system. In~\cite{McSherryCOST}, \citeauthor{McSherryCOST} show that
this metric is frequently misleading when not coupled with performance
comparisons against an appropriate baseline. Providing decreased runtime as core
count increases is certainly a desirable property of a well-engineered parallel
system, but in the end, absolute performance is paramount.

\citeauthor{McSherryCOST} introduce a metric they term COST, or
\textbf{C}onfiguration that \textbf{O}utperforms a \textbf{S}ingle
\textbf{T}hread. This metric measures the amount of hardware required before a
particular system beats the performance of a reasonably designed single threaded
solution. In particular, they found that several parallel data processing
systems from the literature were outperformed by a simple singled threaded
program for various common graph processing tasks, even when the parallel
systems were given over 100x the computational resources.


% but inducing requiring orders of
% magnitude more hardware to achieve the same performance as a single thread is
% very much not.


\section{Results}

\begin{table}
\caption{Example table}
\centering
\begin{tabular}{llr}
\toprule
\multicolumn{2}{c}{Name} \\
\cmidrule(r){1-2}
First name & Last Name & Grade \\
\midrule
John & Doe & $7.5$ \\
Richard & Miles & $2$ \\
\bottomrule
\end{tabular}
\end{table}

\blindtext % Dummy text

\begin{equation}
\label{eq:emc}
e = mc^2
\end{equation}

\blindtext % Dummy text

%----------------------------------------------------------------------------------------
%	REFERENCE LIST
%----------------------------------------------------------------------------------------

\bibliographystyle{unsrtnat}
\bibliography{local}

%----------------------------------------------------------------------------------------

\end{document}
